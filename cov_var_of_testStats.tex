% !TEX TS-program = pdflatex
% !TEX encoding = UTF-8 Unicode

% This is a simple template for a LaTeX document using the "article" class.
% See "book", "report", "letter" for other types of document.

\documentclass[11pt]{article} % use larger type; default would be 10pt

\usepackage[utf8]{inputenc} % set input encoding (not needed with XeLaTeX)
\usepackage{lscape,amsmath,amssymb}
%%% Examples of Article customizations
% These packages are optional, depending whether you want the features they provide.
% See the LaTeX Companion or other references for full information.

%%% PAGE DIMENSIONS
\usepackage{geometry} % to change the page dimensions
\geometry{a4paper} % or letterpaper (US) or a5paper or....
% \geometry{margins=2in} % for example, change the margins to 2 inches all round
% \geometry{landscape} % set up the page for landscape
%   read geometry.pdf for detailed page layout information
\usepackage{multicol,multirow,array}
\usepackage[usenames,dvipsnames]{xcolor}
\usepackage{graphicx} % support the \includegraphics command and options

% \usepackage[parfill]{parskip} % Activate to begin paragraphs with an empty line rather than an indent

%%% PACKAGES
\usepackage{booktabs} % for much better looking tables
\usepackage{array} % for better arrays (eg matrices) in maths
\usepackage{paralist} % very flexible & customisable lists (eg. enumerate/itemize, etc.)
\usepackage{verbatim} % adds environment for commenting out blocks of text & for better verbatim
\usepackage{subfig} % make it possible to include more than one captioned figure/table in a single float
% These packages are all incorporated in the memoir class to one degree or another...

%%% HEADERS & FOOTERS
\usepackage{fancyhdr} % This should be set AFTER setting up the page geometry
\pagestyle{fancy} % options: empty , plain , fancy
\renewcommand{\headrulewidth}{0pt} % customise the layout...
\lhead{}\chead{}\rhead{}
\lfoot{}\cfoot{\thepage}\rfoot{}

%%% SECTION TITLE APPEARANCE
\usepackage{sectsty}
\allsectionsfont{\sffamily\mdseries\upshape} % (See the fntguide.pdf for font help)
% (This matches ConTeXt defaults)

%%% ToC (table of contents) APPEARANCE
\usepackage[nottoc,notlof,notlot]{tocbibind} % Put the bibliography in the ToC
\usepackage[titles,subfigure]{tocloft} % Alter the style of the Table of Contents
\renewcommand{\cftsecfont}{\rmfamily\mdseries\upshape}
\renewcommand{\cftsecpagefont}{\rmfamily\mdseries\upshape} % No bold!

%%% END Article customizations

%%% The "real" document content comes below...

\title{Brief Article}
\author{The Author}
%\date{} % Activate to display a given date or no date (if empty),
         % otherwise the current date is printed 

\newcommand{\snpM}{\mbox{SNP}}
\newcommand{\snpX}{\mbox{SNP}_x}
\newcommand{\snpXM}{\mbox{SNP}_x^M}
\newcommand{\snpXF}{\mbox{SNP}_x^F}


\begin{document}

Consider individual $i$ and subpopulation $k$ and let the estimated ancestry proportion of subpopulation $k$ in individual $i$ on chromosome $c$ to be denoted as $a_{ik}^c$.
We assume that for all $c$ and $c'$, $w_{i,cc'}=w_i$, that is, that the covariance between ancestry at any two pairs of chromosomes within an individual is the same, regardless of which pair of chromosomes we are considering.
Denote 
\begin{align}
cov(a_{ik}^c,a_{ik}^{c'})=w_{i,cc'}=w_i\\
var(a_{ik}^c)=\sigma^2_{ik}\end{align}

First, we find $cov(\sum_{m \notin\{c\}}a_{ik}^{m}, \sum_{m\notin \{c'\}}a_{ik}^{m})$.
In a small example where $c=2$ and $c'=3$ with $m=4$ we see that 
\begin{align}
\sum_{m \notin\{c\}}a_{ik}^{m} = a_1+a_3+a_4  \\
\sum_{m \notin\{c'\}}a_{ik}^{m} = a_1+a_2+a_3
\end{align}
Then it follows that
\begin{align}
cov(\sum_{m \notin\{c\}}a_{ik}^{m}, \sum_{m\notin \{c'\}}a_{ik}^{m}) &=cov(a_1+a_3+a_4, a_1+a_2+a_3)\\
&=cov(a_1,a_1)+cov(a_1,a_2)+cov(a_1,a_3)\\
&+cov(a_3,a_1)+cov(a_3,a_2)+cov(a_3,a_3)\\
&+cov(a_4,a_1)+cov(a_4,a_2)+cov(a_4,a_3)\\
&=2\sigma^2_{ik}+7w_{i}\\
&=(m-2)\sigma^2_{ik}+(m-2)^2w_{i}+(m-1)w_{i} \label{sumsum}
\end{align}
We next find $cov(\sum_{m\notin \{c\}}a_{ik}^{m}, a_{ik}^{c'})$ under the same example where $c=2$, $c'=3$ and $m=4$
\begin{align}
cov(\sum_{m\notin \{c\}}a_{ik}^{m}, a_{ik}^{c'})&=cov(a_1+a_3+a_4,a_3)\\
&=cov(a_1,a_3)+cov(a_3,a_3)+cov(a_4,a_3)\\
&=2w_{i}+\sigma^2_{ik}\\
&=(m-2)w_{i}+\sigma^2_{ik} \label{sumsing}
\end{align}
%We can also find $cov(\sum_{m \notin\{c\}}a_{ik}^{m},\sum_{m \notin\{c\}}a_{ik}^{m})=var(\sum_{m \notin\{c\}}a_{ik}^{m})$.
%\begin{align}
%cov(\sum_{m \notin\{c\}}a_{ik}^{m},\sum_{m \notin\{c\}}a_{ik}^{m}) &= cov(a_1+a_3+a_4,a_1+a_3+a_4)\\
%&=cov(a_1,a_1)+cov(a_1,a_3)+cov(a_1,a_4)\\
%&+cov(a_3,a_1)+cov(a_3,a_3)+cov(a_3,a_4)\\
%&+cov(a_4,a_1)+cov(a_4,a_3)+cov(a_4,a_4)\\
%&=3\sigma^2_{ik}+6w_{i,cc'}\\
%&=(m-1)\sigma^2_{ik}+(m-1)(m-2)w_{i,cc'} \label{varsum}
%\end{align}
%Finally, we calculate $cov(\sum_{m\notin \{c\}}a_{ik}^{m}, a_{ik}^c)$ under the same example where $c=2$ and $m=4$
%\begin{align}
%cov(\sum_{m\notin \{c\}}a_{ik}^{m}, a_{ik}^c)&=cov(a_1+a_3+a_4,a_2)\\
%&=cov(a_1,a_2)+cov(a_3,a_2)+cov(a_4,a_2)\\
%&=3w_{i,cc'}\\
%&=(m-1)w_{i,cc'}  \label{sumsingsame}
%\end{align}

Now we have all the components to find $cov(D_{ik}^c,D_{ik}^{c'})$.
\begin{align}
cov(D_{ik}^c,D_{ik}^{c'}) &= cov(a_{ik}^{-c}-a_{ik}^c,a_{ik}^{-c'}-a_{ik}^{c'})\\
&= cov(a_{ik}^{-c},a_{ik}^{-c'})-cov(a_{ik}^{-c},a_{ik}^{c'})-cov(a_{ik}^c,a_{ik}^{-c'})+cov(a_{ik}^c,a_{ik}^{c'})\\
&= cov(a_{ik}^{-c},a_{ik}^{-c'})-2cov(a_{ik}^{-c},a_{ik}^{c'})+cov(a_{ik}^c,a_{ik}^{c'})\\
&= cov(\frac{1}{22}\sum_{m \notin\{c\}}a_{ik}^{m}, \frac{1}{22}\sum_{m\notin \{c'\}}a_{ik}^{m})-2cov(\frac{1}{22}\sum_{m\notin \{c\}}a_{ik}^{m}, a_{ik}^{c'})+w_{i}\\
&=\Big( \frac{1}{22}\Big) ^2 cov(\sum_{m \notin\{c\}}a_{ik}^{m}, \sum_{m\notin \{c'\}}a_{ik}^{m})-\frac{2}{22}cov(\sum_{m\notin \{c\}}a_{ik}^{m}, a_{ik}^{c'})+w_{i}\\
&=\Big( \frac{1}{22}\Big) ^2\{21(\sigma^2_{ik}+21w_{i})+22w_{i}\} -\frac{2}{22}\{21w_{i}+\sigma^2_{ik}\}+w_{i}\\
&=\frac{1}{(m-1)^2}\{(m-2)\sigma^2_{ik}+(m-2)^2w_{i}+(m-1)w_{i}\} \\
&-\frac{2}{(m-1)}\{(m-2)w_{i}+\sigma^2_{ik}\}+w_{i}
\end{align}
where the first two terms are as found in Equations~\ref{sumsum} and~\ref{sumsing}, respectively and $m=23$.
With some simplification and algebra, this equation becomes 
\begin{equation}
cov(D_{ik}^c,D_{ik}^{c'})=\frac{m}{(m-1)^2}(w_{i}-\sigma^2_{ik})
\end{equation}


%Next, we find $var(D_{ik}^c)=cov(D_{ik}^c,D_{ik}^c)$.
%\begin{align}
%var(D_{ik}^c) &= cov(a_{ik}^{-c}-a_{ik}^c,a_{ik}^{-c}-a_{ik}^c)\\
%&= cov(a_{ik}^{-c},a_{ik}^{-c})-2cov(a_{ik}^{-c},a_{ik}^c)+cov(a_{ik}^c,a_{ik}^c)\\
%&= cov(\frac{1}{22}\sum_{m \notin\{c\}}a_{ik}^{m}, \frac{1}{22}\sum_{m\notin \{c\}}a_{ik}^{m})-2cov(\frac{1}{22}\sum_{m\notin \{c\}}a_{ik}^{m}, a_{ik}^c)+\sigma^2_{ik}\\
%&=\frac{1}{22^2}cov(\sum_{m \notin\{c\}}a_{ik}^{m}, \sum_{m\notin \{c\}}a_{ik}^{m})-\frac{2}{22}cov(\sum_{m\notin \{c\}}a_{ik}^{m}, a_{ik}^c)+\sigma^2_{ik}\\
%&=\frac{1}{22^2}\{(m-1)\sigma^2_{ik}+(m-1)(m-2)w_{i,cc'}\}-\frac{2}{22}(m-1)w_{i,cc'} +\sigma^2_{ik}\\
%&=\frac{1}{(m-1)^2}\{(m-1)\sigma^2_{ik}+(m-1)(m-2)w_{i,cc'}\}-2w_{i,cc'} +\sigma^2_{ik}
%\end{align}
%where the first two terms come from Equations~\ref{varsum} and~\ref{sumsingsame}, respectively and $m=23$.
%With some simplification and algebra, this equation becomes 
%\begin{equation}
%var(D_{ik}^c)=\frac{m}{(m-1)}(\sigma^2_{ik}-w_{i,cc'})
%\end{equation}


Now, we are able to combine these results to calculate the entries in the variance-covariance matrix for our test statistics $T_k^c$.
\begin{align}
cov(T_k^c,T_k^{c'})&=cov(\frac{\overline{D_k^c}}{\sigma_{ck}},\frac{\overline{D_k^{c'}}}{\sigma_{c'k}})\\
&=\frac{1}{\sigma_{ck}\sigma_{c'k}}cov(\overline{D_k^c},\overline{D_k^{c'}})\\
&=\frac{1}{n^2\sigma_{ck}\sigma_{c'k}}cov(D_{1k}^c+\dots+D_{nk}^c,D_{1k}^{c'}+\dots+D_{nk}^{c'})\\
&=\frac{1}{n^2\sigma_{ck}\sigma_{c'k}} \sum_{i=1}^n \frac{m}{(m-1)^2}(w_{i}-\sigma_{i}^2)
\end{align}
The diagonal entries correspond to $var(T_k^c)=1$.

\end{document}




